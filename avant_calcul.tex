\documentclass[]{tufte-handout}

% ams
\usepackage{amssymb,amsmath}

\usepackage{ifxetex,ifluatex}
\usepackage{fixltx2e} % provides \textsubscript
\ifnum 0\ifxetex 1\fi\ifluatex 1\fi=0 % if pdftex
  \usepackage[T1]{fontenc}
  \usepackage[utf8]{inputenc}
\else % if luatex or xelatex
  \makeatletter
  \@ifpackageloaded{fontspec}{}{\usepackage{fontspec}}
  \makeatother
  \defaultfontfeatures{Ligatures=TeX,Scale=MatchLowercase}
  \makeatletter
  \@ifpackageloaded{soul}{
     \renewcommand\allcapsspacing[1]{{\addfontfeature{LetterSpace=15}#1}}
     \renewcommand\smallcapsspacing[1]{{\addfontfeature{LetterSpace=10}#1}}
   }{}
  \makeatother

\fi

% graphix
\usepackage{graphicx}
\setkeys{Gin}{width=\linewidth,totalheight=\textheight,keepaspectratio}

% booktabs
\usepackage{booktabs}

% url
\usepackage{url}

% hyperref
\usepackage{hyperref}

% units.
\usepackage{units}


\setcounter{secnumdepth}{-1}

% citations

% pandoc syntax highlighting

% longtable

% multiplecol
\usepackage{multicol}

% strikeout
\usepackage[normalem]{ulem}

% morefloats
\usepackage{morefloats}


% tightlist macro required by pandoc >= 1.14
\providecommand{\tightlist}{%
  \setlength{\itemsep}{0pt}\setlength{\parskip}{0pt}}

% title / author / date
\title{MIRIS - Calculs préparatoires}
\author{Philippe MICHEL}
\date{}


\begin{document}

\maketitle




\hypertarget{prediction-des-valeurs}{%
\section{Prédiction des valeurs}\label{prediction-des-valeurs}}

\hypertarget{frequence-des-mesures}{%
\section{Fréquence des mesures}\label{frequence-des-mesures}}

\hypertarget{protides}{%
\subsection{Protides}\label{protides}}

La variation prédite est de -0.009 g/jour soit largement inférieure au
seuil de tolérance même sur trois ou cinq jours. À noter, en supprimant
quelques valeurs extrèmes (protides très élevées autour de 5 g/L) on
améliore nettement la courbe de corrélation \& la validité des calculs.
Est-ce que ça a un sens clinique de considérer ces valeurs comme
anormales ?

\hypertarget{lipides}{%
\subsection{Lipides}\label{lipides}}

Pour les lipides la variation prévue est de 0.002 g/jour. De même, cette
variation est trop basse sur quelques jours pour entraîner des erreurs
significatives. Deux mesures par semaine semble être un bon compromis.

\hypertarget{difference-entre-apports-theoriques-et-apports-reels}{%
\section{Différence entre apports théoriques et apports
réels}\label{difference-entre-apports-theoriques-et-apports-reels}}

Sur l'échantillon de mesures on a pour les protides une moyenne de 1.593
± 0.57 g/L. en comptant la moyenne du second échantillon à + 20 \% \&
avec le même écart-type, un risque alpha à 5 \% \& une puissance à 80
\%, chaque groupe doit avoir une taille minimale de 50 échantillons.

Le même calcul pour les lipides donne 139 échantillons poar groupe
(beaucoup plus grande dispersion des mesures).

En comptant les erreurs, oublis etc. on peut estimer des deux groupes de
150 à 200 échantillons sont acceptables \& raisonnables.

De plus en regardant par patient, on voit des différences importantes
de'évolution des taux de protides au fil du temps.

\includegraphics{avant_calcul_files/figure-latex/patients-1}

\hypertarget{annexe}{%
\section{Annexe}\label{annexe}}

Une comparaison rapide des laits crus \& pasteurisés montre des
différences évidentes.

\includegraphics{avant_calcul_files/figure-latex/graph prot-1}

\includegraphics{avant_calcul_files/figure-latex/graphlip-1}



\end{document}
